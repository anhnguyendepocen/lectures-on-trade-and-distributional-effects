\documentclass[pdftex,12pt]{article}
\usepackage[pdftex]{graphicx,color}
\usepackage{setspace}
\usepackage{amsmath,amssymb}
\usepackage{titlesec}
\usepackage{subfigure}
\usepackage{fancyhdr}
\usepackage[longnamesfirst]{natbib}
\usepackage{cite}
\usepackage[paperwidth=8.5in,
left=0.75in,right=0.75in,paperheight=11.0in,textheight=9.50in]{geometry}
\usepackage{xcolor}

\bibliographystyle{ecta}

\definecolor{nblue}{RGB}{0,0,128}

\usepackage[pdftex,bookmarks=false,
pdfdisplaydoctitle = {true},
pdfstartview={XYZ null null 0.65},
pdftitle={CEMFI NOTES: The Distributional Consequences of Trade and Policy Responses.},
pdfauthor={Michael E. Waugh},
pdfkeywords={economics, trade, dynamics, quant econ, consumption, data science, cars, waugh, incomplete markets, inequality, Ricardo, julia, migration, China, trade war, tariffs },
colorlinks=true,linkcolor=darkgray,citecolor=darkgray,urlcolor=darkgray,
breaklinks]{hyperref}


\usepackage{setspace}

\onehalfspace

\renewcommand{\baselinestretch}{1.2}
\renewcommand{\arraystretch}{.7}
\renewcommand\familydefault{\sfdefault}

\titleformat{\section}{\large\bf}{\thesection.}{.5em}{}
\titleformat{\subsection}{\normalsize\bf}{\thesubsection.}{.5em}{}
\titleformat{\subsubsection}{\normalsize\bf}{\thesubsubsection.}{.5em}{}

\def\thesection{\arabic{section}}
\def\thesubsection{\Alph{subsection}}
\def\thesubsubsection{\Roman{subsubsection}}

\newtheorem{proposition}{Proposition}

%\pagestyle{empty}
\lhead{}
\rhead{}
\rfoot{\date}
\cfoot{\thepage}
\lfoot{}
\renewcommand{\headrulewidth}{0pt}

\lfoot{Revised:  \today}

\begin{document}

\pagestyle{fancy}

\noindent \textbf{Notes: \href{http://www.waugheconomics.com/}{ The Distributional Consequences of Trade and Policy Responses.}}\\

\noindent  What are the distributional impacts of trade? And what can policy do to alleviate the negative consequences of trade?  These are the questions that interest me and the goal of this course is to provide a limited introduction to the recent literature studying these issues. By limited, I'm going to provide you my world view of these issues in the literature by organizing it around my own work in: \citet{lyon2019}, \citet{lyon2018redistributing}, and \citet{waugh_consumption}.

\medskip
\noindent While this is a limited view (I only have 3 meetings), I do want to emphasize that their are issues here which are way broader than trade. In fact, some of the most important policy issues that we are facing today regard distributional issues associated with technical change (i.e., robots) and immigration. And what I'm going to talk about relate very closely. Specifically, the empirical techniques used to measure distributional impacts and the modeling approaches to infer the aggregate, welfare, and policy prescriptions are directly applicable to these issues mentioned above. So what I'm saying is if you are looking for a PhD thesis topic/job market paper, these are not bad places to start.

\section{What is the issue?}

\subsection{High-Level}

At a high-level, here is how I think of the issues at hand: In the US, there has been a dramatic rise in inequality over the last 40 years. My take is that for a while the discussion about these issues focused essentially on technological change and the role that it plays. Moreover, inequality/distributional issues were viewed as second-order to issues such as economic growth or cross-county income inequality (Lucas makes this point well). From the a trade perspective, inequality issues were always recognized and a core issue in the literature (H-O model predicts this), but it was hard to find trade driven inequality in the data on top of the problems associated with the H-O model.

\medskip
\noindent I think a couple of things changed within the past 10 years that generated a lot of interest. One is the Great Recession in the US and the observations (particularly of \citet{mian2014explains}) regarding the role that distributional issues played in it. That is some guys had housing wealth that collapsed and these were the guys that bore the brunt of the shock. Second, from the trade perspective, was the work of \citet{david2013china}, who found large distributional effects from China's rise in early 2000's. This is, perhaps, the first time one really saw trade matter with respect to inequality. Overall, one way to put this is the following: \textbf{at the center of the two most important (aggregate) macroeconomic events of the 2000s|rise of China and Great Recession|are distributional issues.}

\medskip
\noindent Also important to reflect on is this: The two key papers here both essentially use the same research design of a local-labor-markets approach. That is they thought about a labor market as being associated with a narrowly defined geography. And then studied how differential exposure of that geography correlates with exposure to China or a credit shock. In this sense, the distributional issues seem strongly related to the spatial distribution of activity.\footnote{I could go on forever about this. At first glance you might think this is natural, but in some sense it's odd. You could still have large distributional effects, but it is within a geography and the local-labor-markets approach would not detect them. Another way to see this is to measure income inequality across county's say in the US. It virtually tracks the change in inequality using individual-level data.} This fact shapes my thinking a lot about these issues as you will see below.

\medskip
\noindent A third issue is technological change. I think the profession has made a lot of progress in working with heterogenous agents (firms, workers, consumers). 10 years ago, there was a lot. Now there is even more. So the point is that we have even better models/computational approaches for dealing with these issues.

\medskip
\noindent Finally, and this is just casual observations, but distributional issues seem first-order to understanding the political change in the US, UK, elsewhere. Hence, economist are revisiting these issues. Forget about trade, just think about the huge explosion of papers talking about competition, market power, labor share, etc. are all motivated by distributional concerns.

\subsection{Low-Level}

As mentioned above, it is empirical observations (the diff-in-diff evidence) that drive the boat here. At a low-level, the research question(s) are really about this: \textbf{While the diff-in-diff evidence are what they are, the key issue is interpretation.} Interpretation is very nuanced for a lot of reasons, I'm going to focus on two:\footnote{As an example of stuff I will omit, I'm not even going to touch or engage the explosion of papers revisiting Bartik and Shift-Share approaches.}
\begin{itemize}
\item \textbf{Level and general equilibrium effects.} While the diff-in-diff evidence is super compelling, it has a huge weakness in that it can never tell you about what is going on with the level (and ge effects). The trade case is really clear. A perfectly plausible interpretation of the \citet{david2013china} evidence is that all labor markets gained, just some gained more than others. Why? The simplest story is through GE where the price of goods is falling making all households better off, even though some labor markets perform relatively worse than others.

\item \textbf{So what? The normative implications.} Again the diff-in-diff is super compelling, but what should a policy maker do? The diff-in-diff evidence tells us nothing! You can't look at \citet{david2013china} and say, well we should not trade with China. And by placing a tariff on Chinese goods will save jobs. Yet the irony is that diff-in-diff evidence is so compelling/interesting it cries out for some kind of investigation into what good policy might look like.
\end{itemize}
A solution to these issues is layer in an economic model to help us better understand the ``real world evidence'' in a controlled environment where we know what is going on. Economic models have weaknesses, i.e. they rely upon strong assumptions about preferences, technology, market behavior. But being explicit about the environment is also a strength, it allows us confront these assumptions and determine if they are important to the model's interpretation of the evidence and then further refine. Second, within an economic model, one can evaluate outcomes associated with different policy. And even think about the best policies (in a formal sense). This is my goal.

\medskip
\noindent A final point that I alluded to above is this: the local-labor-markets approach is predicated on the idea that labor is not moving (at least in the short run). Thus, these research designs are avoiding what is perhaps the \textbf{BIG} question: \textbf{Why is not labor moving to exploit spatial arbitrage opportunities?}

\section{Data, Model, Data}

\subsection{Data}

Per the discussion above, the ``want operator'' is to think about an empirical relationship that takes the form:
\begin{align}
\Delta \log Y_{it} = \beta \Delta \log Z_{it} + \gamma_t + \epsilon_{it}.
\label{eq:adh_data_specification}
\end{align}
where $Y_{it}$ is some outcome in $i$ at $t$ and $Z_{it}$ is some ``shock''. There is a time fixed effect to control for aggregate changes. And everything is done in differences. This is a classic difference in difference specification. The $i$ variable is often geography, something like a county or a commute zone (measure of a local labor market). The shock might be actual trade flows (we can talk about how they do this), tariffs as in my new paper \citet{waugh_consumption}, a measure of trade uncertainty in \citet{pierce2016surprisingly}, etc. Note also this is essentially the same specification that \citet{mian2013household} use in their stuff. It also looks a lot like the stuff that \citet{townsend1994risk} was doing on risk sharing in village India.

Now here are some specific issues that we want to think about:
\begin{itemize}
\item The error term. What is this? What are the issues? Why might there be problems? \citet{david2013china} make a huge deal of this. I found it very hard to think through the issues until I did the stuff below. Often the discussion of the error term often gets overall focused on econometric issues. Fine. \textbf{But from an economic standpoint, it's in the error term that some of the most interesting economic questions lie.}

\item What is $\beta$? You can appeal to some mumbo-jumbo about causal effect, but is this a parameter? A mixture of things (endogenous and exogenous)? If so how does it map into primitives? This steps need to be made when going from the empirical results to normative statements.

\item The aggregate effects. If you are new to this, it might not seem obvious, but all the action is embedded in the time effect. We will see this below.
\end{itemize}
Again, what is the want: I want to understand what \ref{eq:adh_data_specification} is all about. To do so, all I'm going to do is to make a slight, but obviously important departure from the Canonical Ricardian model of trad|add the immobility of labor|and then work through the implications.

\subsection{A Brief Primer on Ricardian Models and \citet{eaton2002technology}}

Here is my quickest introduction.

\medskip
\noindent \textbf{Production.} Within a country, there is a continuum of intermediate goods indexed by $\omega \in [0, 1]$. As in the Ricardian model of \citet{dornbusch1977comparative} and \citet{eaton2002technology}, intermediate goods are not nationally differentiated. Thus, intermediate $\omega$ produced in one country is a perfect substitute for the same intermediate $\omega$ produced by another country. Competitive firms produce intermediate goods with linear production technologies,
\begin{align}
q(\omega) = z(\omega) \ell,
\end{align}
where $z$ is the productivity level of firms and $\ell$ is the number of efficiency units of labor. Intermediate goods are aggregated by a competitive final-goods produce who has a standard CES production function:
\begin{align}
Q = \left[ \int_0^1 q(\omega)^\rho d\omega \right]^{\frac{1}{\rho}},
\label{eq:ces}
\end{align}
where $q(\omega)$ is the quantity of individual intermediate goods $\omega$ demanded by the final-goods firm, and $\rho$ controls the elasticity of substitution across variety, which is $\theta = \frac{1}{1-\rho}$. Transporting intermediate goods across countries is costly. Specifically, consumers and firms face iceberg trade costs when importing and exporting their products. The demand curve in (\ref{eq:ces}) gives rise to the following the demand curve for an individual variety:
\begin{align}
q(\omega) & = \left(\frac{p(\omega)}{P_h}\right)^{-\theta}Q.
\label{eq:demand_curve}
\end{align}
where $Q$ is the aggregate demand for the final good; $P_h$ is the price associated with the final good which will be carried around.

\medskip
\noindent \textbf{Households.} The basic way to think about this is that there is a representative consumer that eats $Q$ and supplies labor inelastically (which we will normalize to one) to \textbf{any labor market within the county.} This is the free-labor mobility assumption that is standard/implicit in Ricardian models of trade.

\medskip
\noindent \textbf{Countries.} To keep things simple, just imagine that there is the home county and a large ROW (which would be a collection of other countries). The way we will do this is the ROW, we will say that there are some prevailing world prices for each commodity $\omega$ which I will denote as $p_w(\omega)$.

\medskip
\noindent \textbf{Sketch of an equilibrium and how trade works.} Ok, so let's just make a couple of observations.

\begin{itemize}
\item The first observation is that within a country, the key equilibrium price is the wage rate $w$. This wage is the same independent of the labor market / goods-producing sector / etc. This derives from the (within-country) free mobility of labor assumption. It's nice, but bad if you want to think about distributional issues.

\item Why is it the key price? Because with free labor mobility (and CRS production function) know that price of any good is independent of who works there, if it's sold, etc. Given competitive firms and that they face the wage rate $w$, we know that their marginal cost (and hence price at which they are willing to sell the good domestically) must be
    \begin{align}
    p(\omega) = \frac{w}{z(\omega)}.
    \label{eq:ek_wage}
    \end{align}
    This is why I love the Ricardian model. Think about this, who can sell at a low price. Well if your wage is low, so this helps China or Bangledesh. Or if your technology is really good, so if $z(\omega)$ is large, this helps the US and Germany. Then the most interesting part is how technology interacts with the wage to determine who trade what, etc.\footnote{This is what distinguishes \citet{eaton2002technology} from saying it is just an application of discrete choice theory.}

\item Who buys what from whom? Second reason why I love the Ricardian model is that really is just an application of the min operator. So buy the good from the country with the lowest price:
    \begin{align}
    \min\left\{ \frac{w}{z(\omega)} , \ \tau p_w(\omega) \right\}
    \end{align}
where the thing on the left is the domestic price, and then the thing on the right is on the world price (inclusive of the trade cost).

\item What I'm going to do is a bit strange, but it will help set up the next model. First, understand this:  If the good is purchased from home, then we know there are workers there, they earn $w$. And how many workers are there? Well enough so that domestic demand (and possible foreign demand, i.e. exports) equals foreign supply, this would be inferred form the demand curve in (\ref{eq:demand_curve}).

    Now let's think of some off equilibrium action. Consider a good $\omega$ where the world is the low cost producer, so the prevailing domestic price is $\tau p_w(\omega)$. If a domestic firm were to produce, it would have to be selling at that price. Ok. If it were to hire workers, the wage it could pay would equal $\tau p_w(\omega) z(\omega)$ or the value of the marginal product of labor. But we know from above that
$\tau p_w(\omega) z(\omega) < w$. And because of free mobility of workers, no workers would accept that wage $\tau p_w(\omega) z(\omega)$.

So the domestic firm producing variety $\omega$ does not operate. No workers are actually ever ``exposed to import competition.'' Again this is all about the free mobility of labor assumption.

\item The distributional assumption to facilitate aggregation. A core insight of \citet{eaton2002technology} is to treat the $z(\omega)$ as a random variable and then exploit the distributional assumptions to aggregate. The second insight is to pick a distribution that can handle the min (or max operator) easily, and that is the Frechet distribution. I won't get into the details about how to derive this. But to see how this works, think about how you would go about using simulation to compute trade flows.
\begin{itemize}
\item Step 1: Assume a distribution for $z$ and $p_w$. Note I'm dropping the $\omega$ subscript as we can treat each good symmetrically given the demand curve associated with \ref{eq:ces}.

\item Draw a bunch of pairs of $z$ and $p_w$.

\item Given a wage, for each pair of $z$ and $p_w$ figure out who is low cost supplier and compute demand, etc.

\item Make sure labor demand equals labor supply. Note in this is the same as ensuring that trade is balanced (it's a static model). If not update the wage.
\end{itemize}
\end{itemize}
What have we learned. Not much. We built intuition for a more complicated setting in which labor is not mobile. As mentioned above, this is a key issue just to even get off the ground and start to think about distributional issues and trade.

\subsection{Immobile Labor $+$ \citet{eaton2002technology} $=$ Static \citet{lyon2019}}

So here is what I'm going to do. I'm going to think about a repeated static world of the model above, but where labor can not move. These I think are the minimal departures from the model above to think about the diff-in-diff approach and evidence.

\medskip
\noindent First, I'm going to say that good-level productivity and world-prices take discrete states and follow a first order Markov chain, with $\mathcal{P}(z'|z)$ being the probability that $z$ transits to $z'$. Similarly for world prices $p_w$. \textbf{ Important! I'm going to assume that the evolution of $z$ and world prices $p_w$ are orthogonal to each other.} Why is this important, wait for it. Now define $\textbf{s}$ as the domestic productivity and world price combination associated with that island, $\textbf{s} = \{z, p_w\}$. We denote $\pi(\textbf{s})$ as the stationary distribution of productivity states and world prices induced by the Markov Chain.

\medskip
\noindent \textbf{Important step!} Denote $\mu(\textbf{s})$ as the measure of households working on an island with state $\textbf{s}$. The key issue is they can only work on islands of state $\textbf{s}$. \textbf{They can not move.} Where does $\mu(\textbf{s})$ come from? I'll layer a model of migration on this do determine it in equilibrium. But for now, just think of it as given. Second, note what this is doing in that what it is doing is tying the product space and the geography together. This is what I call ``company towns'' where a location is completely specialized in a product.

\medskip
\noindent \textbf{Wages.} Competition implies that the wage at which a firm hires labor is:
\begin{align}
w(\textbf{s}) = p(\textbf{s}) z
\label{eq:marginal_products_soe}
\end{align}
or the value of the marginal product of labor. Only at the wage in (\ref{eq:marginal_products_soe}) are intermediate-goods producers willing to produce. Notice how this is different than (\ref{eq:ek_wage}). Now the wage is index by the island specific state $\textbf{s}$! Why labor can not move across locations and try and exploit a wage differential. Instead the household is ``captive'' at wage rate $w(\textbf{s})$.

\medskip
\noindent \textbf{International trade and market clearing.} Like in the \citet{eaton2002technology} world the final-goods producer purchases intermediate goods from the low-cost supplier (it's a still a Ricardian model). But now this decision gives rise to three cases with three different market-clearing conditions: if the good is non-traded; if the good is imported; and if the good is exported and hence different forms of price determination. What's the issue here? Unlike the simple Ricardian model, there is not one wage for which all prices goods prices can be inferred from. Now prices must actually be determined by labor demand and labor supply subject to the option to buy and sell abroad. Below, we describe demand and production in each of these cases.
\begin{itemize}
\item \textbf{Non-traded.} If the good is non-traded, then the domestic price for the home country must satisfy the following inequality: $\frac{p_w}{\tau_{ex}} < p(\textbf{s}) <  \tau_{im} p_w$. That is, from the home country's perspective, it is optimal to source the good domestically and not optimal for the home country to export the product.

    In this case, the market-clearing condition is:
    \begin{align}
\left(\frac{p(\textbf{s})}{P_h}\right)^{-\theta}Q=  z \left( \mu(\textbf{s}) / \pi(\textbf{s}) \right)
\label{eq:non_traded_mc_soe}
    \end{align}
    or that domestic demand equals production. The left-hand part is demand and the right-hand side is supply. That is the the productivity of domestic suppliers multiplied by the supply of labor units in that market.


\item \textbf{Imported.} If the good is imported, then the domestic price for the home country must be $p(\textbf{s}) =  \tau_{im} p_w$. Why? If the price were lower, then it would not be imported. If the domestic price were higher, then the good will be imported with no domestic production and, thus, the prevailing domestic price will equal the imported price. With frictional labor markets, there may be some domestic production so the quantity of imports is
    \begin{align}
 \underbrace{ \left(\left(\frac{\tau_{im} p_w}{P_h}\right)^{-\theta}Q \right) - z\left( \mu(\textbf{s}) / \pi(\textbf{s})\right)}_{\mbox{imports}} \ \ > \ \ 0.
\label{eq:imported_mc_soe}
    \end{align}
    That is home demand (net of home production) is met by imports of the commodity. Rearranging gives
 \begin{align}
 \left(\left(\frac{\tau_{im} p_w}{P_h}\right)^{-\theta}Q \right) \ \ = \ \ z\left( \mu(\textbf{s}) / \pi(\textbf{s})\right) \ \ + \ \  \mbox{imports}(\textbf{s})
\label{eq:imported_mc_soe_im}
 \end{align}
or domestic demand equals domestic production plus imports.


\item \textbf{Exported.} If the good is exported, then the prevailing price must be $p(\textbf{s})\tau_{ex} = p_w$. Why? If the home price were larger, then the good would not be purchased on the world market. And the price can not be lower, as arbitrage implies that the price of the exported good sold in the world market must equal the prevailing price in that market. Finally, note that only the trade cost, not the tariff, matters here. At this price, the quantity of exports is
    \begin{align}
 \underbrace{  \left(\frac{p_w/\tau_{ex}}{P_h}\right)^{-\theta}Q- z\left( \mu(\textbf{s}) / \pi(\textbf{s})\right)}_{- \ \mbox{exports}} \ \ < \ \ 0
 \label{eq:exported_mc_soe}
    \end{align}
    or domestic demand net of production which should be negative, implying that the country is an exporter. Rearranging gives
    \begin{align}
 \left(\frac{p_w/\tau_{ex}}{P_h}\right)^{-\theta}Q \ \ = \ \  z\left( \mu(\textbf{s}) / \pi(\textbf{s})\right)  \ \ - \ \  \mbox{exports}(\textbf{s})
 \label{eq:exported_mc_soe_ex}
    \end{align}
\end{itemize}
or domestic demand equals domestic production minus exports.

\medskip
\noindent \textbf{A sketch of an equilibrium} Hopefully that you see that this is far more intricate than the simple Ricardian model for the reason that an equilibrium is a collection of prices $\{ p(\textbf{s})\}$ for each location that satisfy the demand and supply conditions discussed above. Then for everything to add up a simple condition would be for balanced trade to hold (like in the simple Ricardian model above). To see this last point we can work through the idea that production must equal expenditure. The the value of aggregate production of the final good must equal the value of intermediate goods production
\begin{align}
    Q = \int_{\textbf{s}} p(\textbf{s})z \mu(\textbf{s})
\end{align}
which we can then connect with the expenditure side of GDP through the market clearing conditions for intermediate goods and final goods. Specifically, by connecting the production side with the demand side for non-traded goods in (\ref{eq:non_traded_mc_soe}), imports in (\ref{eq:imported_mc_soe_im}) and exports in  (\ref{eq:exported_mc_soe_ex}) and the equating final demand with consumption we have
\begin{align}
    Q = C +  \int_{\textbf{s}}p(\textbf{s})\mbox{exports}(\textbf{s}) - \int_{\textbf{s}}p(\textbf{s})\mbox{imports}(\textbf{s}).
\label{eq:production_side_gdp}
\end{align}
Or GDP equals consumption (market) plus exports minus imports. Then with balanced trade (again, it's a static model), we get that output will equal what people are eating.

\subsection{Trade}

To illustrate the pattern of trade across islands, first define the following statistic:
\begin{align}
\omega(\textbf{s}) := \frac{p(\textbf{s})z\mu(\textbf{s})}{p(\textbf{s})z\mu(\textbf{s}) \ \ + \ \  p(\textbf{s})\mbox{\normalfont imports}(\textbf{s}) - p(\textbf{s})\mbox{\normalfont exports(\textbf{s})}}.
\label{eq_tax:good_home_share_prp}
\end{align}
What does equation (\ref{eq_tax:good_home_share_prp}) represent? The denominator is the value of domestic consumption: everything domestically produced plus imports minus exports. The numerator is production. The interpretation of (\ref{eq_tax:good_home_share_prp}) is how much of domestic consumption at the island/variety level the home country is producing. This is similar to the micro-level ``home share'' summary statistic emphasized in \citet{arkolakis2012new}. As we discuss below, this statistic (i) provides a clean interpretation of a labor market's exposure to trade and (ii) is tightly connected with local labor market wages.

\medskip
\noindent Figure \ref{fig:home_share} plots the home share (raised to the power of inverse $\theta$) by world price and home productivity. There are three regions to take note of: where goods are imported, exported, and non-traded. First, in the regions where the home share lies below one, demand is greater than supply, and, hence, goods are being imported. This region naturally corresponds to the situation with low world prices or low home productivity|i.e. the economy has a comparative disadvantage in producing these commodities.

\medskip
\noindent Second, in the regions where the home share lies above one, supply is greater than demand, and, hence, goods are being exported. This region corresponds to high world prices or high home productivity. In other words, this is where the country has a comparative advantage and is an exporter of the commodities.

\begin{figure}[p]
\centering{\includegraphics[width=\linewidth]{C:/github/TradeExposure/figures/home_share_smth.pdf}}
\textbf{\caption{Trade: Home Share, $\omega(\textbf{s})^{\frac{1}{\theta}}$ \label{fig:home_share}}}


\centering{\includegraphics[width=\linewidth]{C:/github/TradeExposure/figures/wages_smth.pdf}}
\textbf{\caption{Wages\label{fig:wages_open}}}
\end{figure}

\medskip
\noindent Third, there is the ``table top'' region in the middle, where the home share equals one. Hence, this is the region where the goods are non-traded. Exactly like the inner, non-traded region in the Ricardian model of \citet{dornbusch1977comparative}, the reason is trade costs. In this region, world prices and domestic productivity are not high enough for a producer to be an exporter of these commodities given trade costs. Furthermore, world prices and domestic productivity are not low enough to merit importing these commodities either. Thus, these goods are non-traded.

\subsection{Trade and Wages}

One can connect the pattern of trade across islands/labor markets in Figure (\ref{fig:home_share}) with the structure of wages in the economy. As we show in the Appendix, real wages in a market with state variable $\textbf{s}$ equal
\begin{align}
 w(\textbf{s}) = \omega(\textbf{s})^{\frac{1}{\theta}} \hat \mu( \textbf{s})^{\frac{-1}{\theta}}z^{\frac{\theta-1}{\theta}} Q^{\frac{1}{\theta}}.
 \label{eq_tax:wage_home_share_prp}
\end{align}
Here $\omega(\textbf{s})$ is the home share defined in (\ref{eq_tax:good_home_share_prp}); $\hat \mu( \textbf{s}) = \frac{\mu(\textbf{s})}{\pi(\textbf{s})}$ is the number of labor units;  $z$ is domestic productivity; $C$ is aggregate consumption. Below we walk through the algebraic steps to deriving the relationship discussed above. Note the general insight about this: CES models provide a very tight relationship between expenditure shares and prices, this is what I'm exploiting here.
\begin{enumerate}
\item Begin with the following (generalized) labor demand and supply conditions in units of the final good:
    \begin{align}
p(\textbf{s})\pi(\textbf{s}) \left(\frac{p(\textbf{s})}{P_h}\right)^{-\theta}Q=  p(\textbf{s}) z\mu(\textbf{s}) + p(\textbf{s})\mbox{\normalfont imports}(\textbf{s}) - p(\textbf{s})\mbox{\normalfont exports(\textbf{s})} ,
    \end{align}
where by generalized we mean that the last two terms are only active if the good is imported or exported.

\item Divide both sides by $p(\textbf{s})z\mu(\textbf{s})$ so that we have
    \begin{align}
\frac{p(\textbf{s})\pi(\textbf{s})}{p(\textbf{s})z\mu(\textbf{s})}  \left(\frac{p(\textbf{s})}{P_h}\right)^{-\theta}Q=  \omega(\textbf{s})^{-1}
\label{eq:share_demand_one}
    \end{align}
And now the left hand side is in terms of the home share for the good that the local labor market produces. The key issue is that not all of world prices cancel out and we need to some how connect this with wages. Thus, the next step

\item Then we combine this with how wages are set. Wages equal the value of the marginal product of labor...
\begin{align}
w(\textbf{s}) = \frac{p(\textbf{s}) z}{P_h},
\end{align}
so wages reflect prices. Then if you carefully examine (\ref{eq:marginal_products_soe}) and (\ref{eq:share_demand_one}), you notice that the relationship between wages and prices and how you can substitute one for each other in the demand curve giving.
\begin{align}
\frac{\pi(\textbf{s})}{z\mu(\textbf{s})}  \left(\frac{w(\textbf{s})}{z}\right)^{-\theta}Q=  \omega(\textbf{s})^{-1}
\end{align}

\item Now it is simply about rearranging terms and then substituting in the market clearing condition of $Q$ with consumption $C$ which gives rise to the following:
\begin{align}
 w(\textbf{s}) = \omega(\textbf{s})^{\frac{1}{\theta}} \hat \mu( \textbf{s})^{\frac{-1}{\theta}}z^{\frac{\theta-1}{\theta}} Q^{\frac{1}{\theta}}.
\end{align}
\end{enumerate}
\medskip
\noindent Equation (\ref{eq_tax:wage_home_share_prp}) connects the trade exposure measure in (\ref{eq_tax:good_home_share_prp}) with island-level wages. A smaller home share implies that wages are lower with elasticity $\frac{1}{\theta}$. This means that if imports (relative to domestic production) are larger, then wages in that labor market are lower. Similarly, a larger home share means that wages are higher. While this looks like the ``micro-level'' analog of the aggregate result of \citet{arkolakis2012new} it is different in one important respect: the micro-level wage response to micro-level trade exposure to trade takes the exact opposite sign.

\medskip
\noindent Figure \ref{fig:wages_open} illustrates these observations by plotting the logarithm of pre-tax wages by world price and home productivity so it exactly matches up with Figure \ref{fig:home_share}. As equation (\ref{eq_tax:wage_home_share_prp}) makes clear, there is a tight correspondence between wages and the home share in Figure \ref{fig:home_share}. As in Figure \ref{fig:home_share}, there are three regions to take note of.

\medskip
\noindent The first region is where import competition is prevalent (low world prices or low home productivity) wages are low. A way to understand this result is as follows: wages reflect the value of the marginal product of labor. In import competing islands, trade results in lower prices and, hence, lower wages. The second region is where exporting is prevalent. Exporting regions are able to capture high world prices, and, thus, wages are high in these islands. Finally, the center region is where commodities are non-traded. Here, the gradient of wages very much mimics the increase in domestic productivity. In contrast, where goods are imported or exported, the wage gradient mimics the change in world prices.

\medskip
\noindent \textbf{What have we achieved?} Remember the problem with the standard Ricardian model was that labor mobility killed any way to think about distributional issues. There was one wage. So while Figure \ref{fig:home_share} would look the same in the mobile vs. immobile labor model; Figure \ref{fig:wages_open} would look radically different. In the standard Ricardian model, there would be flat surface which is independent of productivity or wages. In contrast, the immobile labor model delivers a relationship between how much labor is exposed to trade and the wage that labor earns.


\subsection{Back to Data:The ADH Equation}

\medskip
\noindent The preceding results relate closely to the empirical specification and evidence of \citet{david2013china}. To do illustrate the connection, start with (\ref{eq_tax:wage_home_share_prp}) and take log differences across time yielding
\begin{align}
 \Delta \log w(\textbf{s}) =  \underbrace{\frac{1}{\theta}}_{\beta}\Delta \log \left( \omega(\textbf{s})/\hat \mu( \textbf{s})\right) +  \underbrace{\frac{1}{\theta}\Delta \log Q}_{\gamma_t} + \underbrace{\Delta \log \left(z^{\frac{\theta-1}{\theta}}\right)}_{\epsilon_{s,t}},
\label{eq:adh_regression}
\end{align}
which says that the change in wages across locations is summarized by (i) trade exposure via the change in per-worker home share, (ii) the change in aggregate consumption and (iii) the change in location-specific productivity. Equation (\ref{eq:adh_regression}) is the empirical specification in equation (\ref{eq:adh_data_specification}).

\medskip
\noindent Now recall the issues we wanted to think about:
\begin{itemize}
\item \textbf{The error term. What is this? What are the issues?} The error term reflects the unobserved, local productivity shock. This is nice because we have a clear understanding of the problem. There is something confounding the story when we see a labor market with low/declining wages. Is it because the technology is bad? Or is it trade? The model here creates this nice tension! Note this is the classic issue, e.g. steel plans in the US are constantly saying their issues are related to trade, but they also have outdated technology. Which is it?

    Now if absolute advantage was uncorrelated with comparative advantage, then from an econometric stand point (I think things are ok). The problem is that if comparative advantage aligns with absolute advantage (i.e. most productive places are also likely to have a comparative advantage (and this is the natural case and what comes out of \citet{eaton2002technology}), the error term is correlated with a localities trade exposure.  Thus an instrumental variable strategy is necessary to identify the causal effect of trade exposure on wages. \textbf{Deep/non-trivial insight: through Ricardian comparative advantage, the local productivity shock will be correlated with a localities import exposure!}

    The structure of the model suggests several instrumental variable strategies. One valid instrument would be to use the world price (if observed) directly. The world price is orthogonal to domestic productivity (the exclusion restriction), yet correlated with the home trade share. The exclusion restriction follows from our assumption that the stochastic process in for world prices is assumed to be orthogonal to $z$.\footnote{Moveover, the model makes clear that one should be concerned, in general equilibrium, that a change in domestic productivity would feed into world prices and, thus, invalidate this strategy.} An alternative strategy would be to use another country's imports as an instrument. Another country's imports would be orthogonal to the home country's productivity, but correlated with world prices. This, in fact, is quite similar to the instrument proposed in \citet{david2013china}.

\item \textbf{What is $\beta$?} It tells us something about the labor demand curve. Which in this model, labor demand is derived from the CES production function. Hence it tells us something about how substitute products are across space. This is progress because now we can use external evidence to validate this.

    Can you work this out? What if there was a simple labor supply curve in each location? Would that show up in this. What about an employment equation?

\item \textbf{The aggregate effects.} These are all embedded in the time effect. And the way I wrote this done, the only aggregate variable is consumption (if, for example, I was carrying around a TFP term, this would show up here as well. Note, need to do this). Here is why the diff-in-diff stuff is so problematic.  Any change in aggregate trade exposure will also change in aggregate consumption, i.e. the $C$ term. That is all workers benefit from the``aggregate gains to trade'', but the island-level incidence will vary with its trade exposure and may mitigate or completely offset the aggregate benefits from trade. In the static version of this model, this is the key general equilibrium effect. If labor supply were variable, then this would also move aggregate consumption as well.

    If you stare at this long enough, you really start to understand why it is so hard to jump from the coefficient $\beta$, i.e. more trade leads to lower wages, without factoring in what is going on with the aggregate effect.

    A couple of final points? Ok, why not just read off the time effect and... hold on. The example of TFP is relevant here in that the time effect will embed GE forces and exogenous changes like technological change so you can't separate them out. One thought to solve this issue is to use the model which would impose some ``cross-equation'' restriction so that the GE component of the diff-in-diff time effect can be reverse engineered. For example, micro-found labor supply and I'm sure that there would be a time effect showing up as well. Maybe we should work on this.
\end{itemize}



\newpage

\section{The ADH Data}

Open up notebook on github:

\begin{itemize}
\item Aggregate Facts

\item Trade on income and wages

\item Employment effects

\item Social assistance
\end{itemize}


\subsection{Open Issues}

\begin{itemize}
\item \textbf{What is up with manufacturing vs. non-manufacturing?}

\item \textbf{1990s is a Red Herring, 2000s is where the action is.}

\item \textbf{Migration issue}
\end{itemize}

\newpage

\section{Static Model of Migration/Labor Supply}

\medskip
\noindent So I want to connect these results with the literature (specifically \citet{artucc2010trade}, \citet{caliendo2015trade}, \citet{galle2015slicing}) on migration and labor supply and more generally the ``quantitative spatial models'' many people have worked on recently (In \citet{lagakos2013selection} we were essentially the first to kind of work a bunch of this out, others have followed in nice ways like \citet{young2013inequality}, \citet{young2014structural} uses this idea as an alternative explanation for Baumol's Cost Disease; \citet{hsieh2013allocation} study selection's contribution to US aggregate growth; \citet{burstein2015accounting} use similar ideas to measure changes in between-group inequality.)


\medskip
\noindent I'm going to present a static framework along the lines of \citet{caliendo2015trade} (which is a dynamic framework), but then at the end circle back and discuss where the dynamics are important and how they matter. Full disclosure, I have conflicting opinions on these models being (i) these are great models for developing understanding and intuition about the world but (ii) have some questionable implications that I will comeback to at then end.

\medskip
\noindent The basic setup is a Roy model.\footnote{This is a really amazing paper, no equations, nothing, but talks through all the issues about selection, the role of ability in determining things, relative variances, inequality, etc. It kind of lied in obscurity until Heckman and Borjas started actually using it in quantiative/empirical work} That is workers are heterogenous as in that they are endowed with a vector of preferences specific to each island \textbf{s}. In the derivations below, I'm going to think about two types of islands $\textbf{s}_h$ and $\textbf{s}_l$, but this is not important. I will provide the general formulas a bit latter (as practice you might want to derive them your self). A workers utility associated with working in an island is (say, of type $h$) is
\begin{align}
\varepsilon(\textbf{s}_h) w(\textbf{s}_h)
\end{align}
where $\varepsilon_h$ is the workers preference for being in location $h$ and $w(\textbf{s}_h)$ is the wage discussed above. A couple of notes about this formulation. The multiplicative shock may seem strange, but it is essentially the same formulation as \citet{caliendo2015trade}. There, they have log preferences and additive preference shocks with a distributional assumption on the additive shocks which are equivalent to a log transformation of the shocks outline above. A worker then chooses the location that delivers maximal utility.
\begin{align}
\max\{\ \varepsilon(\textbf{s}_h)  w(\textbf{s}_h) \ , \ \varepsilon(\textbf{s}_l)  w(\textbf{s}_l) \ \}
\label{eq:worker_choice}
\end{align}
Note that already you can get a sense of some of the issues with this formulation. When you see somebody in a location, even if wages are bad, or the labor market looks bad, this model says the reason the guy is there is because he has a strong preference for this location. In other words, it's where he wants to be.

\medskip
\noindent \textbf{Timing.} Again, this will be a repeated static model. Each period all states are revealed including the shocks $\varepsilon$ associated with each island type $\textbf{s}$. Then the workers chose where to work given complete information. The next period/day/year things start anew.

\medskip
\noindent Now the want operator here is to connect the individual workers choice in (\ref{eq:worker_choice}) with the measure of workers $\mu(\textbf{s})$ that determine labor supply and production within each location. To do so we will use the \citet{eaton2002technology} insight to chose an well behave distribution on preferences with respect to the max operator above. So across the population of workers, preferences for each island state $\textbf{s}$ are distributed according to independent Frechet distributions:
\begin{align}
G(\varepsilon(\textbf{s})) = \exp\left(-T(\textbf{s})\varepsilon(\textbf{s})^{-\gamma}\right).
\end{align}
The $T(\textbf{s})$ parameter controls the average preference of a worker in island type $\textbf{s}$ and then $\gamma$ controls the dispersion. This is parameterized as in \citet{eaton2002technology} so a larger $\gamma$ means there is less dispersion in preferences. I'm not going to do much with the $T(\textbf{s})$, but one way to think about it is that in reduced form it could represent the idea that some islands with different states are more desirable than others. The common $\gamma$ here is important though because without it the nice math below will not work out.

\subsection{Deriving the Employment Shares}

So we want to derive the employment shares|given an equilibrium wage. Note|for a second I will stop carrying around the $\textbf{s}$. So to compute the share of workers choosing to be in $\textbf{s}_h$, it the boils down to computing the following probability:
\begin{align}
\mathrm{Prob}\left(w_h \varepsilon_h >  w_l\varepsilon_l\right) = \mathrm{Prob}\left(\tilde w \varepsilon_h > \varepsilon_l\right) = \int_0^{\infty}\left[\int_0^{\tilde w\varepsilon_h}g(\varepsilon_h,\varepsilon_l)d\varepsilon_l\right]d\varepsilon_h \label{eq:share_prob}
\end{align}
where $\tilde w$ is the relative wage. Equation \ref{eq:share_prob} says the following: First, fix a $\varepsilon_h$ and compute the probability that $\tilde w \varepsilon_h$ is larger (this is the inside bracket). Second,  integrate over all possible $\varepsilon_h$s (this is the outside bracket). This then computes the probability that a worker selects into a type $h$ island. Furthermore, notice that the inside bracket is just the \href{https://en.wikipedia.org/wiki/Marginal_distribution}{marginal distribution} $G'_{\varepsilon_h}(\varepsilon_h,\tilde w\varepsilon_h)$ evaluated at $\tilde w\varepsilon_h$.

\medskip
\noindent Plugging in the Frechet distribution into Equation (\ref{eq:share_prob}) and evaluating the integral we get
\begin{align}
\mu(\textbf{s}_h) = \mathrm{Prob}\left(w_h \varepsilon_h >  w_l\varepsilon_l\right) = \int_0^{\infty}\left[\int_0^{\tilde w\varepsilon_h}g(\varepsilon_h,\varepsilon_l)d\varepsilon_l\right]d\varepsilon_h = \frac{T_h w_h ^{\gamma}}{T_lw_l^{\gamma} + T_h \tilde w_h ^{\gamma}}
\label{eq:share_formula}
\end{align}
which is the standard share formula. So if workers receive a relatively higher wage in island $h$ , then this means that the share of workers on island type $h$ will be larger. Second, returning to $T$s, this says if island type $h$ is a more ``attractive'' place to be in a utility sense, then more people will live and work there. Finally, note as well how a location is a relative statement. It's never about in absolute terms one place is better than the other, but how one place is relative to the ``average'' location (the object in the denominator) that matters.

\medskip
\noindent Now to match it up with the model discussed above recall that there were a discrete number of states $\textbf{s}$ (the cross of $z$ by $p_w$) which we will denote as $N$. Second, let's not worry about differences in $T$s across island states, so we can simply assume that the $T$s take the same value, one for example. Using the same algebra as above, the general formula is
\begin{align}
\mu(\textbf{s}) = \frac{w(\textbf{s})^{\gamma}}{\sum_{s \in \mathcal{S}} w(\textbf{s})^{\gamma}}
\end{align}
Again, islands with higher wages will have more people working on them. This is what a lot of researchers in trade/spatial literature call the ``labor supply curve.'' It says that as wages in an island type increase, the supply of labor increases with elasticity $\gamma$. A couple of points about this labor supply curve:
\begin{itemize}
\item This setup does nicely pick up the idea that labor will act like it is fixed within a location, but how much labor is fixed there will depend upon the wage in that location. Labor will be in every location, like we started with. Do you know why?

\item The elasticity $\gamma$ plays the following role. As $\gamma$ becomes larger, a given change in wages will translate into a larger change in the share of workers working there. Why? As $\gamma$ becomes larger, the preference shocks become less dispersed. So workers are now more ``sensitive'' to wage considerations when choosing a location. In contrast, as $\gamma$ becomes small, preferences matter more and thus a larger wage response is needed to get people to move.

\item Also, reflect on this. In the standard Ricardian model with free mobility of labor, this is like $\gamma$ becoming arbitrarily large. In other words, the labor supply curve becomes flat with it's intercept at the one, common, equilibrium wage.

\item  Note that this way different than your traditional ``macro'' labor supply discussion. There the issues are either about intensive margin (hours worked) or extensive margin (work or not).\footnote{Adding an extensive margin is easy, just add an island, $w_o$ for being out of the labor force and treating $w_o$ as a transfer or home production (would have to be mindful of market clearing)} Here the margin is really about location or migration. This is a relevant margin that I think the traditional macro labor supply literature has not thought about much. And is an important issue given the decline in migration in the US. But the repeat static nature of things does strain the interpretation here of this being a model of migration.
\end{itemize}
From here we can then setup the system of equations that an equilibrium must satisfy. Specifically, the following (generalized) labor demand and supply conditions in units of the final good:
\begin{align}
\pi(\textbf{s}) \left(\frac{p(\textbf{s})}{P_h}\right)^{-\theta}Q =  z \left(\frac{w(\textbf{s})^{\gamma}}{\sum_{s \in \mathcal{S}} w(\textbf{s})^{\gamma}} \right) + \mbox{\normalfont imports}(\textbf{s}) - \mbox{\normalfont exports(\textbf{s})} ,
\end{align}
where the left-hand side is the total quantity demanded and then the right-hand supply is the quantity supplied. Where the quantity supplied would again depending upon the situation reflect the fact that domestic production may not equal domestic demand and, hence, there are imports or exports.

\newpage

\subsection{A ``Labor Supply'' Equation}

Note that we can use these equations to get towards and ADH style labor supply equation (with caveats). The first thing that I want to do is define this:
\begin{align}
\left(\sum_{s \in \mathcal{S}} w(\textbf{s})^{\gamma}\right)^{\frac{1}{\gamma}} = \Phi.
\label{eq:multi_lat_resit}
\end{align}
which then we can express the labor shares as
\begin{align}
\mu(\textbf{s}) = w(\textbf{s})^{\gamma}\Phi^{-\gamma}
\label{eq:simple_mu}
\end{align}
What is $\Phi$? Lot's of things! But for now the way to think about this is a summary statistic of aggregate, labor market conditions in the sense that it is a geometric sum of individual wages in each island. This is closely related to what \citet{anderson2004trade} called the ``multi-lateral resistance'' term in trade models. Here what it is doing is saying, well where should I locate. It depends not on bilateral comparisons, but on a comparison of each wage vs. the aggregate. And that's what determines the share of people going here. This is also related to the ``irrelevance of independent alternatives'' in standard consumer discrete choice models.

\medskip
\noindent What I'm going to do is to take (\ref{eq_tax:wage_home_share_prp}) and plug it in for the wage in (\ref{eq:simple_mu}). Doing so gives me:
\begin{align}
\mu(\textbf{s}) = \left(\omega(\textbf{s})^{\frac{1}{\theta}} \hat \mu( \textbf{s})^{\frac{-1}{\theta}}z^{\frac{\theta-1}{\theta}} Q^{\frac{1}{\theta}} \right)^{\gamma} \Phi^{-\gamma}
\label{eq:simple_mu}
\end{align}
And then taking first differences across time:
\begin{align}
\Delta \log \mu(\textbf{s}) = \underbrace{\frac{\gamma}{\theta}}_{\beta}\Delta \log \left( \omega(\textbf{s})/\hat \mu( \textbf{s})\right) +  \underbrace{\frac{\gamma}{\theta}\Delta \log Q - \gamma \Delta \log \Phi}_{\gamma_t} + \underbrace{\Delta \log \left(z^{\frac{\gamma (\theta-1)}{\theta}}\right)}_{\epsilon_{s,t}},
\label{eq:adh_LS_regression}
\end{align}
Which is nice. It shows how trade induces labor supply movements. A couple things about this.
\begin{itemize}
\item This equation shares all the nice/bad properties of the ADH specification. GE/Aggregate stuff shows up in the time effect. The error term is correlated with trade exposure, but a plausible instrument would be another countries imports.

\item The new issue here is now that the $\Phi$ is showing up. So what this means is that when we see labor supply adjust, we can just think about determines of the individual island wage, we need to think about how ALL OTHER ISLANDS are changing. This is why the $\Phi$ is in here.

\item This is (I think) essentially what Costas was talking about in his paper last week. In fact, what I think they are doing is using the ideal that the $\Phi$ can be broken up by location and pulled out explicitly, then if you know how all other wages changed, you can determine at least some component of the GE effect which is embedded in $\Phi$.

\item \textbf{Identifying GE effects?} Here is my idea, now we have \ref{eq:adh_LS_regression} and \ref{eq:adh_regression}, so from the two specifications we can pin down the labor demand elasticity $\theta$ and the labor supply elasticity $\gamma$. Then with the two time effects, can we pull out the change in $Q$ and the change in $\Phi$? Need to put in aggregate TFP as a confounder, but perhaps up to a scale we can do this?
\end{itemize}
Where this is not so nice...
\begin{itemize}
\item Again, to call this labor supply, while correct, is not what standard Macro researchers think it is. Fine. The issue is how to map it to the data. If anything, this is really about population or migration, not an extensive margin movement like participation falls within a labor market.

\item If this is really about migration? What was the timing? Is this a good model of migration? I have more questions here than answers, but I have a lot of doubts.

\item If you did take a stand on this as being a model of migration, the issue is that ADH found very small migration effects (zero). \citet{greenland2017import} do find effects once pretrends are taken care of. But I guess what I'm saying this is going to imply probably, a low $\gamma$. In other words a lot of unobserved heterogenaity to prevent people from moving. This is a common issue in these models is that (absent other mechanisims), it's the $\varepsilon$'s that do a lot of the work.
\end{itemize}

\newpage

\subsection{Social Welfare}

Now here is where things start to get (i) disappointing and (ii) still helpful for developing intuition. I'm going to define the following within country (utilitarian) social welfare function:
\begin{align}
U = \sum_{s \in S} \left\{\int \varepsilon(\textbf{s}) w(\textbf{s})d\varepsilon \right\} \pi(\textbf{s})
\label{eq:expected_utility}
\end{align}
where starting from the most inner part: utility for a worker on island $\textbf{s}$, then integrated over all workers in choosing to work in island $\textbf{s}$ (I need better notation here), then integrated over all islands (this is the sum with the probability weights). So the key issue is to compute the inner bracket, that is utility conditional on choosing that sector.

\medskip
\noindent To compute expected utility conditional on choosing an island, we need to derive the preference distribution|conditional on those that select into the island. (Again to simplify notation, I'm going to revert back to the two state case and work through it, then show the general formula). This probability density is\ldots
\begin{align}
\mathrm{Prob}\left( \varepsilon_h < \varepsilon | w_h \varepsilon_h >  w_l\varepsilon_l\right) = \frac{\mathrm{Prob}\left( \varepsilon_h < \varepsilon , w_h \varepsilon_h >  w_l\varepsilon_l\right)}{\mathrm{Prob}\left(w_h \varepsilon_h >  w_l\varepsilon_l\right)}
\end{align}
where the right hand side is the probability that a workers productivity is less than some value, conditional on being in that island and the left hand side is the definition of this conditional density. To work through this, lets first compute the probability in the numerator
\begin{align}
\mathrm{Prob}\left( \varepsilon_h < \varepsilon , \tilde w \varepsilon_h > \varepsilon_l\right) = \int_0^{\varepsilon}\left[\int_0^{\tilde w\varepsilon_h}g(\varepsilon_h,\varepsilon_l)d\varepsilon_l\right]d\varepsilon_h.
\end{align}
This is the probability that a worker productivity for the $h$ island has productivity below $\varepsilon$ and chooses to work with an $h$ type island. How to you compute this probability, first fix an $\tilde w\varepsilon_h$ and compute the probability that the $\varepsilon_l$ is less than that value. This is what the integral in the bracket computes. Then sum up over the possible $\varepsilon_h$s up to the value $\varepsilon$. Like before, this observation also implies that the probability in the brackets is simply the \href{https://en.wikipedia.org/wiki/Marginal_distribution}{marginal distribution} $G'_{\varepsilon_h}(\varepsilon_h, \tilde w\varepsilon_h)$ evaluated at $\tilde w\varepsilon_h$. The probability in the denominator is closely related\ldots
\begin{align}
\mathrm{Prob}\left(\tilde w \varepsilon_h > \varepsilon_l\right) = \int_0^{\infty}\left[\int_0^{\tilde w\varepsilon_h}g(\varepsilon_h,\varepsilon_l)d\varepsilon_l\right]d\varepsilon_h
\end{align}
Which just integrates over all possible $\varepsilon_h$s to compute the probability that a worker selects into a type $h$ firm. Furthermore, this stuff in the brackets is also just the \href{https://en.wikipedia.org/wiki/Marginal_distribution}{marginal distribution} $G'_{\varepsilon_h}(\varepsilon_h,\tilde w\varepsilon_h)$ evaluated at $w\varepsilon_h$. Together, this implies that
\begin{align}
\displaystyle
\mathrm{Prob}\left( \varepsilon_h < \varepsilon | w_h \varepsilon_h >  w_l\varepsilon_l\right) = \int_0^{\varepsilon}G'_{\varepsilon_h}(\varepsilon_h,\tilde w\varepsilon_h)d\varepsilon_h \ \ \bigg/ \ \ \int_0^{\infty}G'_{\varepsilon_h}(\varepsilon_h,\tilde w\varepsilon_h)d\varepsilon_h
\end{align}
Now how do we compute these integrals? So the marginal distribution $G'_{\varepsilon_h}(\varepsilon_h, \tilde w\varepsilon_h)$ is $g_{\varepsilon_h}(\varepsilon_h)G_{\varepsilon_l}(\tilde w\varepsilon_h)$ which follows from the independence of the draws. Then we just evaluate these integrals.

\medskip
\noindent The denominator of this probability is the same as the share formula so
\begin{align}
\int_0^{\infty}G'_{\varepsilon_h}(\varepsilon_h,w\varepsilon_h)d\varepsilon_h = \frac{T_h \tilde w ^{\gamma}}{T_l + T_h \tilde w ^{\gamma}}
\end{align}
Then the numerator of this probability is
\begin{align}
\int_0^{\varepsilon}G'_{\varepsilon_h}(\varepsilon_h,\tilde \varepsilon_h)d\varepsilon_h = \mu_h \exp\left[-\left(T_l\tilde w ^{-\gamma} + T_h \right)\varepsilon^{-\gamma}\right]
\end{align}
which then implies that the ratio of the two is equal to
\begin{align}
\mathrm{Prob}\left( \varepsilon_h < \varepsilon | w_h \varepsilon_h >  w_l\varepsilon_l\right) = \exp\left[-\left(T_l\tilde w ^{-\gamma} + T_h \right)\varepsilon^{-\gamma}\right]
\end{align}
which says that the conditional probability is itself distributed Frechet with centering parameter $\left(T_l\tilde w ^{-\gamma} + T_h \right)$.

\medskip
\noindent To make things a bit more simple, notice that $\left(T_l\tilde w ^{-\gamma} + T_h \right) = T_h \mu_h^{-1}$ in (\ref{eq:share_formula}). This follow by multiplying and dividing the high share formula by $\tilde w^{-\gamma}$. Then to compute the conditional expectation we just simply compute the expected value associated with the distribution
\begin{align}
\mathrm{Prob}\left( \varepsilon_h < \varepsilon | w_h \varepsilon_h >  w_l\varepsilon_l\right) = \exp\left[-T_h\mu_h^{-1}\varepsilon^{-\gamma}\right]
\end{align}
which implies|after abstracting from constant terms| the expected preference value a worker receives is
\begin{align}
\mathrm{E}\left(\varepsilon_h\ |\varepsilon_h w_h > \varepsilon_l w_l\right) = T_h^{\frac{1}{\gamma}} \mu_h^{\frac{-1}{\gamma}}
\end{align}
which has the flavor of the \citet{arkolakis2012new} formula. The key thing to notice is that as the share of workers becomes small, then average preference value increases. This is classic selection at work in the sense that only those who value a location the most will remain there. The general relationship takes the same form
\begin{align}
\mathrm{E}\left(\varepsilon(\textbf{s}) \ |\varepsilon(\textbf{s}) w(\textbf{s}) > \max_{s \in S}\left\{\varepsilon(\textbf{s}) w(\textbf{s})\right\}\right) = T(\textbf{s})^{\frac{1}{\gamma}} \mu(\textbf{s})^{\frac{-1}{\gamma}}
\end{align}
Now for simplicity, let's drop the $T$s and then ask what is utility within a group? It is the wage times the average preference for that location
\begin{align}
w(\textbf{s}) \mu(\textbf{s})^{\frac{-1}{\gamma}} =  w(\textbf{s}) \left(\frac{w(\textbf{s})^{\gamma}}{\sum_{s \in \mathcal{S}} w(\textbf{s})^{\gamma}}\right)^{\frac{-1}{\gamma}}
 =  \left(\sum_{s \in \mathcal{S}} w(\textbf{s})^{\gamma}\right)^{\frac{1}{\gamma}} = \Phi.
\label{eq:expected_utility_phi}
\end{align}
Notice what just happened. \textbf{Utility is independent of the island a worker is on. It does not depend on \textbf{s}}. It says, in terms of distributional effects, even though labor earnings are moving around with trade exposure (\ref{eq:adh_regression}), there is no effect on utility in the sense that one place gains more or less. It's all the same. Another way to put this is suppose we did have group utility $\Phi$ and then regressed the change in this on the change in trade exposure, there would be no effect!

\medskip
\noindent Let me try and explain what is going on here and then raise some open questions. First, this is purely about the distributional assumption (independent Frechet). If we had log normal on the preferences, for example, you would not get this result. From an economics standpoint we need to think through the selection issues. Take for example some place $\textbf{s}$ that experiences a decline in the wage. What happens, well there are some marginal guys in $\textbf{s}$, that will chose a different location. Who are these guys? Well they were the ones with the comparative preference for that location. But in this model comparative preference lines up with absolute preference, i.e. the guys who are there, in a likelihood sense are also the (absolute terms) the high preference guys. But the guys who select out probably had the lowest of the high for that location. So that location is shedding the guys with the lowest preference in that location. What does that mean, average utility is increasing.

\medskip
\noindent Bear with me a bit more. So while the wage might be decreasing, only the highest guys remain. So how average utility behaves in the group is an open question. With the Frechet it's not. The extensive margin movement behaves in a way so that average utility across locations is always equalized!\footnote{In \citet{lagakos2013selection} we had the Frechet shocks on individual productivity and you get the result that average wages across locations are equalized! So even though there is all this selection going on in the background/micro-level, the aggregate outcome looks like homogenous labor model.} This is deep. Not obvious. But it is central to the mechanics about how Frechet trade models or Frechet Roy models operate.

A couple of things I want to point out/raise questions
\begin{itemize}
\item Is wage inequality good here or bad? $\gamma > 1$ and I forgot jensens inequality. But you tell me?

    Update, it looks like a mean preserving spread in wages will \textbf{increase utility}! I think this is all about option value affects. Because of the max operator on this, what people really want is a good wage out there that can help them.

    What would be interesting here is to work through (\ref{eq:expected_utility_phi}) and make a couple of decompositions: (i) pull out the level effect from a lower price of goods (so the CES) component and (ii) the ``spread'' component. A second interesting decomposition is relative to the case of frictionless labor mobility. So characterize the wage, call it $w*$, then compare this relative to the value.

\item The right measure of utility. While the computations above are correct. What is not obvious to me is the correct way to compare things across regimes. This I think depends on the nature of the preference shocks. Are they fixed? Or do they change over time? The related issue is do we want to do group by group or literally individual level. The real problem is that I've been very loose about a lot of thing (also an explicit dynamic model would solve most of this)

    Here is the issue, for example, take a infra-marginal guy in location $\textbf{s}$ that experiences a wage loss. He is not moving (assuming the preference shocks are drawn once at the begging of time) because he is infra-marginal in that location. He has to take a hit and be worse of than before. Right?\footnote{The only hesitation is that, in the not uncommon Frechet magic, the structure of wages won't change and this guy will be indifferent or will win, this can't be right, there would be no labor supply response?}

\item Now here is another dimension in which I'm struggling. Overall, the question is would the government want to intervene. First instinct is no. Second instinct is that from behind the veil perhaps some kind of insurance might want to be provided. Like the infra-marginal guy who happens to be in the hit location would like help. What I'm thinking here is some kind of tax and transfer policy. Sounds plausible, but then back to the first bullet point, policies that reduce wage inequality would seem to lower welfare.

\end{itemize}

\section{Importance of a Dynamic Model}

Here let me raise a couple of issues why a dynamic model is important in this context. One of the want's is to say, ok, China shows up big time in the 2000s, what happens to the people living in locations that relatively hardest hit\ldots
\begin{itemize}
\item To answer this question, we need the following (i) a notion of initial conditions and where people are initially are (the theory above does this) and (ii) a notion of where  people and the economy are going in response to the China/Trade shock. The theory above is|at best|vague about this issue. At worst, it has absolutely nothing to saw.

\item I think a dynamic model fill in (ii). Thinking about individual decisions and linking there actions across time allows us to think through/answer questions about how the economy would respond to a China shock.

\item A third issue is that a dynamic model also allows us to bring in data about how people are moving across space. In the static model, this is not clear. And again in the dynamic migration model, this becomes much more obvious. And then it allows us to validate the model in a more concrete sense.
\end{itemize}

\section{Just Preference Shocks, Dynamic Model}

So the stuff I had above is a lot like \citet{galle2015slicing}. The stuff below will be much more like \citet{caliendo2015trade} and \citet{artucc2010trade}. Also here at CEMFI (or was) Joan Monras did work on this as well, these are essentially dynamic discrete choice models.

\medskip
\noindent Here is how I'm going to do this (it will be a bit different, but I'll show how to collapse this thing to \citet{caliendo2015trade} and \citet{artucc2010trade}) later. The value function of a guy who stays on an island will look like this.
\begin{align}
V(\textbf{s})^{st} = w(\textbf{s}) + \beta E V(\textbf{s}') + \epsilon^{st}
\label{eq:expected_utility_stay}
\end{align}
so he gets to utility associated with the wage that he earns, he receives a preference shock (now additive) associated with staying, then he gets next periods expected value function. Note that the state of an island is moving around with the productivity shocks and world price shocks. That's why we have the expectation here (the idea is like I'm in a good location, but stuff happens now it's like a bad location).\footnote{This source of randomness is not in your typical migration model. I want it here for a lot of reasons to be explained below}. Then the guy who moves gets this
\begin{align}
V(\textbf{s})^{mv} = w(\textbf{s}) + \beta \tilde V + \epsilon^{mv} - \tau^{mv}
\label{eq:expected_utility_move}
\end{align}
so I get the wage and what I'm going to do is say that $\tilde V$ is the expected value associated with a move (which I will describe a bit more below). Then these guys get a moving shock which is shifted down by some $\tau^{mv}$. The latter value is a dis-utility of moving and it is here to make sure people are not flying all over the place. Also while utility is linear (so what I'm about to say does not matter), this is usually talked about in terms of this being in units of utils. That is there is no sense in which there is a resource cost of a move and it won't be as it will not show up in the equilibrium conditions above). 

\medskip
\noindent The expected value associated with a move $\tilde V$. Here we can do a lot of things. I'm going to do the simplest. When you move you go to a random island. Specifically, 
\begin{align}
\tilde V = \sum_{s \in S} \pi(\textbf{s}) V(\textbf{s})
\end{align}
where you get the value function (defined below) of a guy on island $\textbf{s}$ and the probability distribution across this is according to the invariant distribution associated with the Markov chain described above. This is not the only way to do this. You could also do something like this: Once you move, you could target the island with the highest wage, call this island $s*$ and then 
\begin{align}
\tilde V = \sum_{s \in S} \mathcal{P}(\textbf{s} | \textbf{s*}) V(\textbf{s})
\end{align}
where notice, because there is randomness in the island type, you can't target perfectly. It's like saying, today New York is awesome. Ok I'm going there. I show up next period and things changed. New York is not as great as it was. Well what is the probability that New York becomes like, say, Detroit\ldots it's given by the Markov chain $\mathcal{P}(\textbf{s} | \textbf{s*})$!

\medskip
\noindent One more example, you could add in another layer of preference shocks. So there is $\varepsilon$ for each location $\textbf{s}$ like above. How would you do this? 
\begin{align}
\textbf{s*} = \mbox{arg}\max_{s \in S} \left\{V(\textbf{s}) + \varepsilon(\textbf{s}) \right\}\\
\nonumber \\
\tilde V = \sum_{s \in S} \mathcal{P}(\textbf{s} | \textbf{s*}) \left( V(\textbf{s}) + \varepsilon(\textbf{s}) \right)
\end{align}
Same idea as with the max labor market, not it's about the preference shock shows up as well. If we remove the uncertainty on islands, then we collapse to \citet{caliendo2015trade} and \citet{artucc2010trade}. 



\medskip
\noindent I still have not told you about the value function. Then the value function is the outer envelope of the two move or stay value functions described above:
\begin{align}
V(\textbf{s}) = \max \left\{ V(\textbf{s})^{st}, V(\textbf{s})^{mv} \right\}
\end{align}
which then if we say that the $\epsilon$s are distributed Type 1 extreme value distribution with scale parameter $\sigma$  (like in the Frechet case, bigger $\sigma$ means less variance) we get the standard share formula which says that the fraction of people who stay on island of type $\textbf{s}$ is
\begin{align}
\hat \mu(\textbf{s})^{st} = \frac{\exp \left( V(\textbf{s})^{st} \right)^{\sigma}}{\exp \left(V(\textbf{s})^{st}\right)^{\sigma} + \exp \left(V(\textbf{s})^{mv}\right)^{\sigma}}
\label{eq:migration_shares}
\end{align}
so the key thing here is (i) we get a similar, nice closed for expression for how many people are staying or moving (and unlike the static model, they are staying or moving) and then (ii) the key difference is that it's not the wage that matters but the value associated with being in that location. Also note that $\mu(\textbf{s})^{st}$ is not the population on the island. It is the fraction of people that move. As we talk about below, what we need to do is to compute the distribution of people across islands. 

\medskip
\noindent \textbf{A sketch of an equilibrium} This is a bit more complicated because of two reasons: (i) the value functions need to be computed using dynamic programming techniques (value function iteration will work well here) and (ii) we need to be mindful about what the population is on each island type. This latter point is tricky because there are two things going on. First, people are leaving islands and second, island's types are changing. That is the shocks are moving around and the state is transiting. Let's walk through these steps:
\begin{itemize}
\item Step one. Guess some wages.

\item Step two. Use value function iteration to compute the (\ref{eq:expected_utility_stay}) and (\ref{eq:expected_utility_move}).

\item Step three. Use (\ref{eq:migration_shares}) to compute who is staying and who is moving.

\item Step four. \textbf{Hard part.} Compute the stationary distribution of people across islands. Here is how this would work in a simple two state (closed economy case). So there $\textbf{s}_1$ and $\textbf{s}_2$ with a Markov chain that determines the probability that a state transits from one to another, so $\mathcal{P}(\textbf{s}' | \textbf{s})$. Also remember that our assumption is that movers go to a random island! 

    Here is what we want to think about is what happens to the population of people on island of type state $\textbf{s}_1$? So some people are not moving which is $\hat \mu(\textbf{s}_1)^{st}$, but remember islands are changing type, so that only a fraction of those guys are still in type one areas how many
    \begin{align}
    \mu(\textbf{s}_1) = \underbrace{\mathcal{P}(\textbf{s}_{1} | \textbf{s}_{1}) \times \mu(\textbf{s}_1) \times \hat \mu(\textbf{s}_1)^{st} + \mathcal{P}(\textbf{s}_1 | \textbf{s}_{2})\times \mu(\textbf{s}_2) \times \hat \mu(\textbf{s}_2)^{st}}_{\mbox{stayers}}
    \nonumber \\
    \nonumber + \underbrace{\pi(\textbf{s}_1)\times (\mu(\textbf{s}_1) \times \mu(\textbf{s}_1)^{mv} + \mu(\textbf{s}_2)\times \mu(\textbf{s}_2)^{mv})}_{\mbox{movers}}
    \end{align}
    This says the following. The mass on type one islands $\mu(\textbf{s}_1)$:
    \begin{itemize}
    \item $\mathcal{P}(\textbf{s}_{1} | \textbf{s}_{1}) \times \mu(\textbf{s}_1) \times \hat \mu(\textbf{s}_1)^{st}$ probability that islands stay type one, times the mass on type one, times the stayers. 
    \item $\mathcal{P}(\textbf{s}_1 | \textbf{s}_{2})\times \mu(\textbf{s}_2) \times \hat \mu(\textbf{s}_2)^{st}$ these are the stayers from island type two which has some probability of becoming a type one island.
    \item Then there are some movers: $\pi(\textbf{s}_1)\times (\mu(\textbf{s}_1) \times \mu(\textbf{s}_1)^{mv}$ these are like the ``failed movers'' in the sense that they moved, but ended back up on type on islands. And then $\pi(\textbf{s}_1) \mu(\textbf{s}_1)\times \mu(\textbf{s}_2)^{mv})$ are the guys coming from type two islands.    
    \end{itemize}
    
    
    Or to put this in matrix form we would have:
    \begin{align*}
    \begin{bmatrix}\mu(\textbf{s}_1) \\ \vdots \\ \mu(\textbf{s}_N)\end{bmatrix} =
    \begin{bmatrix}\mathcal{P}(\textbf{s}_{1} | \textbf{s}_{1})& \ldots & \mathcal{P}(\textbf{s}_{1} | \textbf{s}_{N}) \\ \vdots & \ldots & \vdots \\
    \mathcal{P}(\textbf{s}_{1} | \textbf{s}_{N})& \ldots & \mathcal{P}(\textbf{s}_{N} | \textbf{s}_{N})\end{bmatrix}
    \times \begin{bmatrix}\mu(\textbf{s}_1) \times \hat \mu(\textbf{s}_1)^{st} \\ \vdots \\ \mu(\textbf{s}_1) \times \hat \mu(\textbf{s}_N)^{st})\end{bmatrix} +
    \begin{bmatrix}\pi(\textbf{s}_1) \\ \vdots \\ \pi(\textbf{s}_N)\end{bmatrix}\times \begin{bmatrix} \sum_{N}\mu(\textbf{s}_n) \times \mu(\textbf{s}_n)^{mv} \\ \vdots \\ \sum_{N}\mu(\textbf{s}_n) \times \mu(\textbf{s}_n)^{mv} \end{bmatrix}
    \end{align*}
    This is why this is hard is that now we have a system of equations with the populations of $\mu$ on both sides. A stationary equilibrium is a $\mu$ that satisfies this. One approach (which should work as long as the markov chain is well behaved) is to guess a $\mu$ and then iterate on this until it converges.

\item Step Five. Given the populations, check if the wages that satisfy the system of equations above holds. If not, then update the wages (or prices) and do it all again.
\end{itemize}
Lots of things to note about this setting:
\begin{itemize}
\item \textbf{The Moving Protocal.} The idea here is that if you change the moving protocol, all you need todo is swap out the last term. In fact, if you (i) eliminate the random shocks and (ii) have prefernce shocks across locations driving where people end up, then you would essentially get the same exact formula from \citet{caliendo2015trade} which is like a fixed point on the sum over the moving probabilities. 
    
\item \textbf{Where do people end up?} This is not as clear as in the static model, but here is some intuition. First, what is important/nice is that the preference moving shocks ensure that people are always moving. Then with the moving protocol, people are always ending up in all locations. So in other words, there are very strong dispersion forces built into the shock structure that get people ``spread'' across space. This is important because a worry would be (from a computational perspective) that there is (i) some absorbing state that makes no sense or (ii) multiple invariant distributions (and hence equilibria) that depend upon initial conditions. I don't have a proof of this, but computation it is clear that there is one stationary distribution and the shock structure helps a lot here. 
    
    Ok fine. Still you will get the idea that more people will tend to end up in high wage locations, just like the static model. But what keeps them there?
    
\item \textbf{Why do people move?} Obviously it has to do with the preference shock. But (i) current wage and (ii) future wages. The current wage is clear, in high wage locations people are less likely to move. But the Markov chain starts to play a role in the sense that the persistence of an island state will start to matter. Suppose that stuff is iid. Well if you are in a bad island, there is a lot of mean reversion, so it makes less sense to suffer the utility cost of a move and go some place else. If stuff is near a random walk (so a lot of persistence), then this will be a strong force to exit if things get bad, why? Because things are going to be bad for a long time.
    
    I think this is an important force (especially if there is curvature on utility and the temporal patter of consumption matters). Moving is like an asset, but an asset that pays off very well in bad states of the world.  
    
    Back to where people end up, what happens with mild to strong persistence in the shocks is that (i) if you are in a bad place you try and get out and (ii) if you are in a good place you stay for a while. The in equilibrium wages have to move in such a way that will support these desires. 
    
\item \textbf{Why don't people move?} One is the preference shock. Two is the disutility of moving. But also here there is some uncertainty associated with the move. The moving protocol matters here because there is a chance I end up in a similarly bad location. With the linear utility, I don't think this force is super strong. But with curvature, then this will start to bite. 
    
\item \textbf{Gains/Losses from trade?} So in a lot of ways, the wage issues discussed above are exactly the same! All we did is swap out the ``labor supply side'' block of the model and put in a dynamic formulation. This is nice. So people who are initially in a bad or become import exposed will suffer wage losses as in the ADH sense. 
    
    But here is the thing. This is a dynamic model. So stuff in the future is baked into utility today. And recall what I was mentioning above about people (eventually go all over the place). So they dynamic model introduces a new gain from trade in the sense that at some point down the road, losers today become winners in the future. This is either because (i) they can move or (ii) islands change. This is what \citet{artucc2010trade} emphasized as option value effects. Yes it's bad, but in the future the options are much more attractive. This kind of also relates to the static model is that there is a sense in which variance in wages across islands is good, why? Because (i) the high wage is really good and (ii) the low wage is not as bad as I can always move.
  
\end{itemize}
Final example to think through and develop intuition. Suppose there are two types of guys high $\beta^h$ and low $\beta^l$, so some guys are more patient then others. I want to think of these as the young and old. Where will they end up in the stationary equilibrium? I have not done this, but here is my conjecture:
\begin{itemize}
\item Bad places are full of impatient guys. Why, migration is a dynamic decision, and the old/impatient guys care less about the future. So they are less likely to try and escape from the bad islands.

\item Now if comparative advantage aligns with absolute advantage...bad places, with old people, are going to be hit by an increase in import exposure. 

\item What about the gains from trade here. Well bad places, with old people suffer. Now here is the thing, because of dynamics, while you might be hurt today, you benefit from the high wages that come with trade in other places.  But the old guys are impatient. These things don't matter much to them. So they are doubly hurt. First, because of their impatience they are likely to get hit. And second, they get none of the dynamic gains that arrive in this model.
\end{itemize}
\medskip
\noindent \textbf{Social welfare and Normative Issues.} First, with the linear utility, this should be equivalent to a complete-markets allocation. And no obvious role for intervention. 

\medskip
\noindent Once we introduce curvature, now there are risks that matter (location and moving risk) that the guys would like insurance against. And notice that once trade changes in the nature of wages across islands, it is also changing the risk of location and moving. Now we have an economy that (i) is not first-best because of the absence of markets to insure against the risks described above and (ii) trade is changing the desire and importance of this risk. This is where the paper that Spencer and I wrote \citet{lyon2018redistributing} which things about this exact tension.

\medskip
\noindent 



\newpage

\section{Welfare and Normative Issues}





\newpage

\small
\bibliography{micro_price_bibtex}

\end{document}
